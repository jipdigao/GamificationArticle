
%Document generated using DScaffolding from https://www.mindmeister.com/1191114871
\documentclass{article}
\usepackage[utf8]{inputenc}

\newcommand{\todo}[1] {\iffalse #1 \fi} %Use \todo{} command for bringing ideas back to the mind map

\title{Gamification}
\author{}

\begin{document}

\maketitle
      

\section{Introduction}

%Describe the practice in which the problem addressed appears

    
%Describe the practical problem addressed, its significance and its causes
A major problem is that decreasing number of participants. This problem is of particular concern as it is now well established that it can lead to low impact \cite{Hamari2014} \cite{Hamari2014} \cite{Hamari2014} \cite{Hamari2014}. Causes can be diverse: (1) nice banquet \cite{Hamari2014} and (2) cause 2 \cite{Liu2017}.
\todo{que no se nos olvide considerar las causa 5}
    
\textbf{ Moreover, causa 4 has been shown to be related to this problem. }%Summarise existing research including knowledge gaps and give an account for similar and/or alternative solutions to the problem

    
%Formulate goals and present Kernel theories used as a basis for the artefact design

    
%Describe the kind of artefact that is developed or evaluated
This article presents a novel artefact
    
%Formulate research questions
tarari que te vi \todo{poner bibliografia}

    
%Summarize the contributions and their significance

      
%Overview of the research strategies and methods used
This article has followed a Design Science Research approach.

%Describe the structure of the paper
The remainder of the paper is structured as follows: 

%Optional - illustrate the relevance and significance of the problem with an example
    
      
\bibliographystyle{unsrt}
\bibliography{references}

\end{document}
    
